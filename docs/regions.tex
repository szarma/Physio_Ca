\documentclass[a4paper,11pt,final]{article}
\usepackage{fancyvrb, color, graphicx, hyperref, amsmath, url}
%\usepackage{palatino}
\usepackage{pygments}
\usepackage[a4paper,text={16.5cm,25.2cm},centering]{geometry}

\hypersetup{
  pdfauthor = {Srdjan Sarikas},
  pdftitle={Intracellular Receptors},
  colorlinks=TRUE,
  linkcolor=black,
  citecolor=crimson,
  urlcolor=navy
}

%\setlength{\parindent}{0pt}
%\setlength{\parskip}{1.2ex}

\title{Intracellular Receptors \\ {\small Materials and Methods: Analysis and Processing}}

%\author{}
\date{\today}

\begin{document}
\maketitle

%<<echo=False>>=
%import matplotlib.pyplot as plt
%import numpy as np
%import pandas as pd
%from islets import load_regions
%@

A typical experiment involving imaging of pancreatic slices in our lab concerns a single field of view (FOV)
showing 10s-100s of cells, in a recording of at least several, often dozens, of gigabytes.
Current tools (ImageJ, \dots) rely on loading the recording, or its part, into memory, for viewing, analysis, and processing.
It also requires laborious and long human engagement.
We have developed a set of interdependent tools to automatize as much as possible the pipeline.
The crucial elements of our are the following:
\begin{itemize}
\item (Semi-)automatic detection of regions of interest (ROIs);
\item Transformation of ROI time traces into standard score ("z-score");
\item Quantification of the phenotype for each ROI in terms of the distribution of events of different durations.
\end{itemize}

\begin{figure}[h]
\centering
\includegraphics[width=17cm]{figures/pipeline.pdf}
\label{fig:pipeline}
\caption{
An illustration of our processing and analysis pipeline:
{\it (i)}  From a full movie ($T{\times}x{\times}y$), we calculate the mean (or other statistic) across all frames.
{\it (ii)} We pass the mean image through a band-pass filter and define ROIs by detecting local peaks.
{\it (iii)} We save ROIs with all the important information (time traces, ROI coordinates, movie statistics, recording frequency, pixel size, etc.).
{\it (iv)} Traces contain features at very different time scales---with different timescales presumably important for different cell types. We collect them into separable events for analysis.
}
\end{figure}

\section{(Semi-)Automatic Detection of Regions of Interest}

%<<caption="Fig.1">>=
%plt.plot(range(3))
%@


\end{document}
